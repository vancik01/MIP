% Metódy inžinierskej práce

\documentclass[10pt,twoside,slovak,a4paper]{article}

\usepackage[slovak]{babel}
\usepackage{pdfpages}
%\usepackage[T1]{fontenc}
\usepackage[IL2]{fontenc} % lepšia sadzba písmena Ľ než v T1
\usepackage[utf8]{inputenc}
\usepackage{graphicx}
\usepackage{url} % príkaz \url na formátovanie URL
\usepackage{hyperref} % odkazy v texte budú aktívne (pri niektorých triedach dokumentov spôsobuje posun textu)

\usepackage{cite}
%\usepackage{times}

\pagestyle{headings}

\title{Kvízová hra, ako moderný predajný nástroj\thanks{Semestrálny projekt v predmete Metódy inžinierskej práce, ak. rok 2022/23, vedenie: XY}} 


\author{Martin Vančo\\[2pt]
	{\small Slovenská technická univerzita v Bratislave}\\
	{\small Fakulta informatiky a informačných technológií}\\
	{\small \texttt{xvancom@stuba.sk}}
}


\date{\small 28. september 2022} % upravte



\begin{document}

\maketitle

\begin{abstract}
	Témou môjho výstupného projektu z predmetu Metódy inžinierskej práce bude spojenie gamifikácie a marketingu, konkrétne využitie kvízových hier na zvýšenie interakcie medzi zákazníkom a predávajúcou firmou. Gamifikáciou sa rozumie využitie herných mechaník v odvetviach bežného života, napríklad pri učení,
	v zdravotníctve, či v priemysle. Marketing je oblasť skúmajúca dopyt a ponuku produktu. Jej cieľom je následne zabezpečiť, aby bol produkt pre zákazníka čo najatraktívnejší, čoho výsledkom by malo byť zvýšenie celkového predaja, či nárast povedomia o predávajúcej firme. Mojim cieľom bude prepojiť tieto dve odvetvia a ukázať, akým spôsobom dokáže kvízová hra zvýšiť predaj, či záujem o konkrétny produkt na trhu.
\ldots
\end{abstract}


\section{Úvod}
S príchodom sociálnych médií zažíva marketing obrovský boom. Vďaka platenej reklame na internete nebolo nájdenie potenciálnych koncových zákazníkov nikdy jednoduchšie. S príchodom novej éry sociálnych médií  je čoraz náročnejšie svoje publikum udržať. Pre udržanie nášho publika je veľmi dôležité vytvárať relevantný a zaujímavý obsah týkajúci sa produktu alebo značky, a následne ho publikovať na webe či sociálnych sieťach. 

Jedným z typov obsahu, ktorý dokáže zaujať naše publikum sú rôzne súťaže, kvízy či prieskumy. Tento obsah využíva prvky gamifikácie. V tomto článku sa zaoberá simbióze týchto dvoch oblastí.

\section{Definícia marketingu}\label{Definícia marketingu}

%Z obr.~\ref{f:rozhod} je všetko jasné. 

\begin{figure*}[tbh]
\centering
%\includegraphics[scale=1.0]{diagram.pdf}
Aj text môže byť prezentovaný ako obrázok. Stane sa z neho označný plávajúci objekt. Po vytvorení diagramu zrušte znak \texttt{\%} pred príkazom \verb|\includegraphics| označte tento riadok ako komentár (tiež pomocou znaku \texttt{\%}).
\caption{Rozhodujúci argument.}
\label{f:rozhod}
\end{figure*}



\section{Iná časť} \label{ina}

Základným problémom je teda\ldots{} Najprv sa pozrieme na nejaké vysvetlenie (časť~\ref{ina:nejake}), a potom na ešte nejaké (časť~\ref{ina:nejake}).\footnote{Niekedy môžete potrebovať aj poznámku pod čiarou.}

Môže sa zdať, že problém vlastne nejestvuje\cite{Coplien:MPD}, ale bolo dokázané, že to tak nie je~\cite{Czarnecki:Staged, Czarnecki:Progress}. Napriek tomu, aj dnes na webe narazíme na všelijaké pochybné názory\cite{PLP-Framework}. Dôležité veci možno \emph{zdôrazniť kurzívou}.


\subsection{Nejaké vysvetlenie} \label{ina:nejake}


Niekedy treba uviesť zoznam:

\begin{itemize}
\item jedna vec
\item druhá vec
\item druhá vec
\item druhá vec
	\begin{itemize}
	\item x
	\item y
	\end{itemize}
\end{itemize}

Ten istý zoznam, len číslovaný:

\begin{enumerate}
\item jedna vec
\item druhá vec
	\begin{enumerate}
	\item x
	\item y
	\end{enumerate}
\end{enumerate}


\subsection{Ešte nejaké vysvetlenie} \label{ina:este}

\paragraph{Veľmi dôležitá poznámka.}
Niekedy je potrebné nadpisom označiť odsek. Text pokračuje hneď za nadpisom.



\section{Dôležitá časť} \label{dolezita}


\section{Ešte dôležitejšia časť} \label{dolezitejsia}




\section{Záver} \label{zaver} % prípadne iný variant názvu



%\acknowledgement{Ak niekomu chcete poďakovať\ldots}


% týmto sa generuje zoznam literatúry z obsahu súboru literatura.bib podľa toho, na čo sa v článku odkazujete
\bibliography{literatura}
\bibliographystyle{plain} % prípadne alpha, abbrv alebo hociktorý iný
\end{document}
